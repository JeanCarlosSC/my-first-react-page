Factores de conversión de unidades
Longitud
1 m = 100 cm = 1000 mm = 10^6 \mu m = 10^9 nm
1 km = 1000 m = 0.6214 mi
1 in = 2.540 cm

Unidades
1 J = 1 kg \cdot m^2 / s^2

Sears Zemansky - Física universitaria
1.5 Incertidumbre y cifras significativas

Multiplicación o división: El resultado no debe tener más cifras significativas que el número inicial con menos cifras significativas:
\frac{0.745 \cdot 2.2}{3.885} = 0.42
1.32578 \cdot 10^7 \cdot 4.11 \cdot 10^{-3} = 5.45 \cdot 10^4

Suma o resta: El número de cifras significativas se determina por el número inicial con mayor incertidumbre (es decir, el menor número de dígitos a la derecha del punto decimal):
27.156 + 138.2 - 11.74 = 153.6

Constantes numéricas
Constantes físicas fundamentales
Rapidez de la luz en el vacío
    Símbolo: c
    valor: 2.99792458 \cdot 10^8 m/s